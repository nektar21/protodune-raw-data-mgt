The Offline Computing in protoDUNE includes two principal domains:
\begin{itemize}
\item ``Express streams'' to be run on the CERN cluster, for fast calibrations and QA purposes
\item Production and analysis on other sites, in Europe, the United States and elsewhere
\end{itemize}

\noindent
Development and management of the code to be used in these two cases will be done largely by the same group of people responsible for offline software in DUNE.

At the time of writing, only general parameters of the protoDUNE Offline Computing Model can be determined. For example, it is fairly certain that the data management system for offline computing will be based on the same infrastructure as the raw data management system as described throughout this document.

As to Workload Management (meaning assigning particular tasks or groups of tasks to Grid sites or any other suitable compute elements, and coordinating that with the data management), there are a few options currently under consideration, including the ``jobsub’’ toolkit provided by FNAL, and a few of more comprehensive Workload Management Systems that need to be evaluated for the needs of protoDUNE.
