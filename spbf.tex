\documentclass[pdftex,12pt,letter]{article}
\usepackage[margin=0.75in]{geometry}
\usepackage{verbatim}
\usepackage{graphicx}
\usepackage{cite}
\usepackage[pdftex,pdfpagelabels,bookmarks,hyperindex,hyperfigures]{hyperref}

\bibliographystyle{unsrt}

\newcommand{\fixme}[1]{\textbf{FIXME: #1}}    


\title{Single-Phase protoDUNE Disk Buffer Farm}
\date{\today}
\author{M. Potekhin and B. Viren}

\begin{document}
\maketitle

\begin{abstract}
  The requirements for a data buffer between the single-phase
  protoDUNE detector DAQ and central CERN computing are given.  A
  model of the DAQ and expected CERN environment is developed and
  emulated at Brookhaven National Lab in order to test and develop
  possible designs for a buffer system based on XRootD.  A proposal
  for a design to be built at CERN is presented.  Finally, a
  functional test of the Fermilab File Transfer System is described
  and results are presented.
\end{abstract}

\tableofcontents

\pagebreak


\section{Overview}

We will fill in this document as we do the work.  Right now it is just
an outline.

\section{Requirements and Assumptions}

This section will give:
\begin{itemize}
\item data rates based on the data taking ``scenarios'' spread sheet
\item CERN requirements (eg, 3 days buffer)
\item assumptions about various bandwidths (eg network and disk)
\item ...
\end{itemize}


\section{DAQ and Environment Model}

We will develop a simple, functionally equivalent emulation for the SP
DAQ data source and implement a model of the network environment from
DAQ to EOS.

\section{Results}

We will look at:

\begin{itemize}
\item scaling as a function of number of concurrent writes 
\item forced different bottlenecks at NIC and disk.
\item effect of simultaneous reads/writes to same box, bus, disk. 
\item ...
\end{itemize}

This test will be done at BNL, probably on existing RACF LBNE nodes.
For it we have identified 3 interactive nodes and 7 pure-batch nodes
with an existing round-robin XRootD installation.

\section{Design}

Based on results this section will provide 
\begin{itemize}
\item design for hardware (number of nodes, disk speeds, RAM, etc)
\item configuration
\item any limitations (eg, on number of simultaneous reads+writes)
\end{itemize}

\section{FTS Testing}

There are three parts to this testing.

\begin{enumerate}
\item Buffer farm $\to$ ``EOS''
\item ``EOS'' $\to$ FNAL
\item FNAL $\to$ ``OSG''
\end{enumerate}

The quotes are used as ``EOS'' will be emulated with a simple storage
node at BNL and ``OSG'' will simply again be nodes at BNL.  The goal
is a functional tests of all these types of transfers if not a
performance one.

More information about FTS is in the document ``Design of the Data
Management System for the protoDUNE Experiment'').

\section{References}
\bibliography{citedb}


\end{document}

%%% Local Variables:
%%% mode: latex
%%% TeX-master: t
%%% End:
