\subsection{Requirements}
The following is a summary of basic requirements for the protoDUNE data management system:
\begin{itemize}

\item Transfer raw data files from both online disk buffer farms of the DP and SP prototype detectors (NP02 and NP04 respectively) to CERN EOS disk and from there to CERN tape (CASTOR), FNAL tape (Enstore) and other end-points.

\item Ensure that the throughput is adequate and there are no bottlenecks for the Data Acquisition System given the expected data rates over the nominal two month running (see table)

\item Record metadata about file status and outcome of file operations

\item Operate at CERN and FNAL with support for initial setup and ongoing operations
\item Provide monitoring of overall system health, alerts on error and support debugging of problems.

\item Provide triggers to perform file operations (copy, delete) based on configurable rules

\item Support “express lane” process at CERN and other institutions.

\end{itemize}

\subsection{Data Characteristics}
The following table presents various characteristics of the data itself and of the data transmission system driven by the extreme
of each DP and SP detector and which the file handling system must accommodate on the assumption that they apply to both detectors.

\begin{table}[ht!]
	\centering
	\begin{tabular}{p{3.0in}p{0.95in}}
	\hline
	Total raw data & 2.5\,PB \\

	Total number of raw files & 2\,M \\

	Sustained data rate & 20\,Gbps \\

	Max latency to reach EOS & 10\,min \\

	Max latency to reach express processing & 10\,min\\

	Max simultaneous files in FTS dropbox & 1\,M\\

	Max simultaneous active files in transfer & 50,000\\

	Max file registrations & 200,000\,day$^{-1}$ \\


	\hline
\end{tabular}
\caption{protoDUNE data characteristics}
\label{tab:pdunedatachar}
\end{table}






