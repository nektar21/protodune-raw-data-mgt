\documentclass[pdftex,12pt,letter]{article}
\usepackage[margin=0.75in]{geometry}
\usepackage{verbatim}
\usepackage{graphicx}
\usepackage{xspace}
\usepackage{cite}
\usepackage{url}
\usepackage[pdftex,pdfpagelabels,bookmarks,hyperindex,hyperfigures]{hyperref}

\newcommand{\pd}{protoDUNE\xspace}
\newcommand{\xrd}{XRootD\xspace}

\title{The clustered storage option for the protoDUNE NP04 Online Buffer}
\date{\today}
\author{N. Benekos, M. Potekhin and B. Viren}


\begin{document}
\maketitle

\begin{abstract}
\noindent  This note describes the clustered storage
solution for the online buffer of the Single-Phase \pd (experiment NP04).
Basic data characteristics and quantitative parameters of such storage are estimated. \xrd is proposed as the underlying
storage clustering technology. A portion of the existing   \textit{Neutrino Platform} computer cluster at CERN
can potentially be utilized for the development, testing and  implementation of the NP04 online buffer.  
\end{abstract}

\section{Data Characteristics}

The current ``data scenario'' estimates \cite{docdb1086} bracket the expected range of data rate
and data volume due to in-spill beam triggers and out-of-spill cosmic ray muon triggers.
For historical reasons, the scenarios at the ends of this range are named ``Central'' and ``High rate''.

Two driving assumptions are the beam trigger rate and that one
cosmic-ray muon trigger is acquired out-of-spill for every in-spill
beam trigger.  Below is the summary of principal data characteristics:

\begin{description}
\item[trigger rate] 25 -- 50 Hz
\item[peak data rate (DAQ internal)] 1.5 -- 3.0 GByte/sec (instantaneous during spill)
\item[daily data volume] 25 -- 50TB
\item[3-day buffer capacity] 150 -- 300TB
\end{description}

\section{Concept}

The \texttt{neut} cluster is now being formed using some 300 nodes in
total reclaimed from ATLAS.  We will dedicate about 50 nodes in
support of developing the disk buffer system for the single-phase
protoDUNE detector adequately scaled for storing three days worth of
expected data.  To label this sub-cluster we say
``\texttt{neut-spbuf}''.  Initially \texttt{neut-spbuf} will be for
developing and testing the buffer system design.  Meanwhile, we will
explore what is needed to migrate \texttt{neut-spbuf} into actual
operation.

\section{Disk}

The current \texttt{neut} nodes have very limited disk storage.  The
\texttt{neut-spbuf} nodes must be upgraded to provide storage to meet
the 3-day buffer requirement. To meet the ``High rate'' requirement we
will install $2\times 3$TB SATA disks in each of the 50
\texttt{neut-spbuf} nodes.

\section{Networking}

The ``High rate'' scenario requires sinking a peak of 3.0 GByte/sec
(24 Gbps) throughput during the beam spill.  Between spills, when
cosmic muon triggers are acquired, the throughput is somewhat reduced
but we take 3.0 GByte/sec as our requirement.  Spread across the 50
\texttt{neut-spbuf} nodes these streams this will approximately fill
50\% of the existing 1Gbps NICs.  We expect similar multiplicity at
the data production end (the Event Builder layer of the pD/SP DAQ).

During initial testing we will request a 20 Gbps link between the
current location of \texttt{neut}\footnote{CERN building 185} and
central CERN computing including EOS and the pD/SP detector
site\footnote{CERN building EHN1}.

To supply this connectivity we require 50 switch ports at 1Gbps and
(effectively) one switch port at 20Gbps.  Based on our current design
it is possible to segment the network streams so that the total
bandwidth is spread over multiple switches, for example two switches
each with 25 ports at 1Gbps and 1 port with 10 Gbps.  One example
switch is the Cisco SG500X-48P which can provide 48 1Gbps ports and
ample ports on the high-bandwidth side.  One such switch is needed on
the DAQ end of the 20Gbps link and one on the \texttt{neut-spbuf} end.

\begin{thebibliography}{1}

\bibitem{docdb1086}
{DUNE DocDB 1086: \textit{ protoDUNE/SP data scenarios with full stream (spreadsheet)}}\\
\url{http://docs.dunescience.org:8080/cgi-bin/ShowDocument?docid=1086}

\bibitem{docdb1209}
{DUNE DocDB 1209: \textit{Design of the Data Management System for the protoDUNE Experiment}}\\
\url{http://docs.dunescience.org:8080/cgi-bin/ShowDocument?docid=1209}

\bibitem{docdb1212}
{DUNE DocDB 1212: \textit{Basic Requirements for the protoDUNE Raw Data Mangement System}}\\
\url{http://docs.dunescience.org:8080/cgi-bin/ShowDocument?docid=1212}


\bibitem{xrootd}
{XRootD, high performance, scalable fault tolerant access to data  repositories}.\\
  \url{http://xrootd.org/}.

\end{thebibliography}


\end{document}

%%% Local Variables:
%%% mode: latex
%%% TeX-master: t
%%% End:
